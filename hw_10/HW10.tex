\documentclass[11pt]{article}
%\parskip 1.0\parskip plus 3pt minus 1pt
\renewcommand{\baselinestretch}{1.5}

\setlength{\oddsidemargin}{-0.1in}
\setlength{\evensidemargin}{-0.1in} 
\setlength{\textwidth}{6.54in}
\setlength{\topmargin}{0in} 
\setlength{\textheight}{8.5in}
%\setlength{\oddsidemargin}{0in}
%\setlength{\evensidemargin}{0.15in} 
%\setlength{\textwidth}{6.2in}
%\setlength{\topmargin}{-0.3in} 
%\setlength{\textheight}{8.9in}

\linespread{1}

\usepackage{makeidx}
\usepackage{amsmath,amssymb, amsthm}
\usepackage{latexsym,remreset}
\usepackage[toc,page]{appendix}
\usepackage{graphicx}
\usepackage{multirow}
\usepackage{bbm}
%\usepackage[pdftex]{graphicx} 
\usepackage[usenames]{color}



\usepackage{url}
%% Define a new 'leo' style for the package that will use a smaller font.
\makeatletter
\def\url@leostyle{%
  \@ifundefined{selectfont}{\def\UrlFont{\sf}}{\def\UrlFont{\small\ttfamily}}}
\makeatother
%% Now actually use the newly defined style.
\urlstyle{leo}

\newtheorem{assumption}{Assumption}
\newtheorem{definition}{Definition}
\newtheorem{lemma}{Lemma}
\newtheorem{proposition}{Proposition}
\newtheorem{corollary}{Corollary}
\newtheorem{remark}{Remark}
\newtheorem{theorem}{Theorem}
\newcommand{\mb}[1]{\ensuremath{\boldsymbol{#1}}}
\newcommand{\vol}{{\sf volume}}
\newcommand{\Stochb}{$\Pi_{\mathrm {Stoch}}(b)$}
\newcommand{\tf}{\text{translation factor}}


\title{ISyE6669-OAN Homework Week 10}
\author{Fall 2021}
\date{}

\begin{document}

\maketitle

\section*{Column Generation}
In this problem, we want to walk you through the column generation algorithm to solve the cutting stock problem. Consider the following formulation

\begin{align*}
\min \quad & \sum_{j=1}^n x_i \\
\text{s.t.}\quad & \sum_{j=1}^N \mb{A}_j x_j = \mb{b} \\
& x_j \geq 0, \qquad \forall j=1,\dots, N.
\end{align*}

The problem has the following data. Customers need three types of smaller widths: $w_1 = 7, w_2 = 11, w_3 = 16$ with quantities $b_1 = 15, b_2 = 30, b_3 = 20$. The width of a big roll is $W = 80$. 
\begin{enumerate}
%\item Write down the restricted master problem, assuming you have picked a subset $I$ of initial patterns. 
\item Assume the column generation algorithm starts from the following initial patterns:
\begin{align*}
\mb{A}_1 = \begin{bmatrix} 10 \\ 0 \\ 0\end{bmatrix}, \mb{A}_2 = \begin{bmatrix} 0 \\ 7 \\ 0\end{bmatrix}, \mb{A}_3 = \begin{bmatrix} 0 \\ 0 \\5\end{bmatrix}.
\end{align*}
Write down the restricted master problem (RMP) using these patterns. Solve this RMP by hand. Find the optimal basis $\mb{B}$ and its inverse $\mb{B}^{-1}$. Find the optimal dual solution $\hat{\mb{y}}^\top=\mb{c}_B^\top\mb{B}^{-1}$.


\color{black}
\item Solve RMP in Python CVX. Write down the optimal solution, the optimal basis
$\mb{B}$, and its inverse $\mb{B}^{-1}$. Find the optimal dual solution $\hat{\mb{y}}^\top=\mb{c}_B^\top\mb{B}^{-1}$. To take the inverse
of $\mb{B}$, you can use a calculator or computer program. In this iteration, you should be able to solve this LP by hand. But we ask you to set up the code in CVX and solve it using CVX. This code will be used in later iterations.



\color{black}
\item Write down the pricing problem, i.e. the knapsack problem using the above data and the optimal dual solution you found. 
%\item Write a small program in Xpress to solve the above knapsack problem. This code can be also useful for Project 1. Turn in a hard copy of your code.


\color{black}
\item Solve the pricing problem in CVX. Should we terminate the column generation algorithm at this point? Explain. If the column generation should continue, what is the new pattern generated by the pricing problem? 



\color{black}
\item If the column generation should continue, then augment (RMP) with the new column and solve it in CVX again. You can easily modify your CVX code by incorporating the new column. Write down the optimal solution, the optimal basis $\mb{B}$, and the inverse $\mb{B}^{-1}$. Compute the dual
variable. Then solve the pricing problem again by modifying the date in your code. Should you terminate the column generation at this iteration? Explain. If the column generation should continue, do the same for all the following iterations until the column generation terminates.

\color{black}
\item Write down the final optimal solution, the optimal basis, and the optimal objective value.

\color{black}
\item For this problem, you need to submit all your codes for all the steps separately. Name them as question1 RMP step1, question1 Pricing step1, question1 RMP step2, and so on.


\end{enumerate}


\end{document}